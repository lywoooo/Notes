% ==================== PACKAGE IMPORTS ====================
% Mathematics
\usepackage{amsmath,amssymb,amsfonts,amsthm}
\usepackage{derivative}

% Graphics and Figures
\usepackage{graphicx}
\usepackage{tikz}
\usetikzlibrary{intersections, angles, quotes, positioning, arrows.meta}
\usepackage{pgfplots}
\pgfplotsset{compat=1.18}

% Layout and Formatting
\usepackage[margin=3cm]{geometry}
\usepackage{fancyhdr}
\usepackage{float}
\usepackage{multicol}
\usepackage{multirow}

% Tables and Arrays
\usepackage{array}
\usepackage{tabularx}
\usepackage[table, dvipsnames]{xcolor}

% Lists
\usepackage{enumitem}

% Units and Symbols
\usepackage{textcomp, gensymb}
\usepackage[per-mode=symbol]{siunitx}

% Utilities
\usepackage{etoolbox}
\usepackage{ifthen}
\usepackage{xstring}
\usepackage{pgffor}

% Colored Boxes
\usepackage[many]{tcolorbox}
\usepackage{thmtools}
\usepackage[framemethod=TikZ]{mdframed}

% Algorithms
\usepackage[linesnumbered,lined,vlined,ruled,commentsnumbered,resetcount,algochapter]{algorithm2e}

% Polynomials
\usepackage{polynom}

% Hyperlinks (load last)
\usepackage[colorlinks=true,linkcolor=cyan,urlcolor=magenta,citecolor=violet]{hyperref}
\urlstyle{same}

% ==================== PDF SETTINGS ====================
\pdfsuppresswarningpagegroup=1

% ==================== PAGE STYLE ====================
\pagestyle{fancy}
\fancyhf{}
\fancyhead[L]{\leftmark}
\fancyfoot[C]{\thepage}

% ==================== COLOR DEFINITIONS ====================
\definecolor{correct}{HTML}{009900}

% ==================== CUSTOM COMMANDS ====================
% Correction markup
\newcommand\correct[2]{\ensuremath{\:}{\color{red}{#1}}\ensuremath{\to}{\color{correct}{#2}}\ensuremath{\:}}
\newcommand\green[1]{{\color{correct}{#1}}}

% Derivatives
\newcommand{\ddx}{\frac{d}{dx}}
\newcommand{\ddxd}{\frac{\mathrm{d}}{\mathrm{d}x}}
\newcommand{\dydx}{\frac{dy}{dx}}

% ==================== TIKZ STYLES ====================
\tikzset{
	force/.style={thick, {Circle[length=2pt]}-stealth, shorten <=-1pt}
}

% ==================== ALGORITHM SETTINGS ====================
\SetKwComment{Comment}{// }{}
\SetArgSty{textsl}
\def\algocflineautorefname{Algorithm}

% Clear item autoref name
\makeatletter
\def\itemautorefname{\@gobble}
\makeatother

% ==================== THEOREM STYLES ====================
\mdfsetup{skipabove=1em,skipbelow=0em}

% Green box (Definitions)
\declaretheoremstyle[
	headfont=\bfseries\sffamily\color{ForestGreen!70!black},
	bodyfont=\normalfont,
	mdframed={
		linewidth=2pt,
		rightline=false, topline=false, bottomline=false,
		linecolor=ForestGreen, backgroundcolor=ForestGreen!5,
		nobreak=false
	}
]{thmgreenbox}

\declaretheoremstyle[
	headfont=\bfseries\sffamily\color{ForestGreen!70!black},
	bodyfont=\normalfont,
	mdframed={
		linewidth=2pt,
		rightline=false, topline=false, bottomline=false,
		linecolor=ForestGreen, backgroundcolor=ForestGreen!8,
		nobreak=false
	}
]{thmgreen2box}

\declaretheoremstyle[
	headfont=\bfseries\sffamily\color{SeaGreen!70!black},
	bodyfont=\normalfont,
	mdframed={
		linewidth=2pt,
		rightline=false, topline=false, bottomline=false,
		linecolor=SeaGreen, backgroundcolor=SeaGreen!2,
		nobreak=false
	}
]{thmgreen3box}

% Blue box (Examples, Claims)
\declaretheoremstyle[
	headfont=\bfseries\sffamily\color{NavyBlue!70!black},
	bodyfont=\normalfont,
	mdframed={
		linewidth=2pt,
		rightline=false, topline=false, bottomline=false,
		linecolor=NavyBlue, backgroundcolor=NavyBlue!5,
		nobreak=false
	}
]{thmbluebox}

\declaretheoremstyle[
	headfont=\bfseries\sffamily\color{MidnightBlue!70!black},
	bodyfont=\normalfont,
	mdframed={
		linewidth=2pt,
		rightline=false, topline=false, bottomline=false,
		linecolor=MidnightBlue, backgroundcolor=MidnightBlue!5,
		nobreak=false
	}
]{thmblue2box}

\declaretheoremstyle[
	headfont=\bfseries\sffamily\color{TealBlue!70!black},
	bodyfont=\normalfont,
	mdframed={
		linewidth=2pt,
		rightline=false, topline=false, bottomline=false,
		linecolor=TealBlue,
		nobreak=false
	}
]{thmblueline}

% Red box (Theorems, Lemmas, Propositions)
\declaretheoremstyle[
	headfont=\bfseries\sffamily\color{RawSienna!70!black},
	bodyfont=\normalfont,
	mdframed={
		linewidth=2pt,
		rightline=false, topline=false, bottomline=false,
		linecolor=RawSienna, backgroundcolor=RawSienna!5,
		nobreak=false
	}
]{thmredbox}

\declaretheoremstyle[
	headfont=\bfseries\sffamily\color{RawSienna!70!black},
	bodyfont=\normalfont,
	mdframed={
		linewidth=2pt,
		rightline=false, topline=false, bottomline=false,
		linecolor=RawSienna, backgroundcolor=RawSienna!8,
		nobreak=false
	}
]{thmred2box}

% Pink box (Problems, Exercises)
\declaretheoremstyle[
	headfont=\bfseries\sffamily\color{WildStrawberry!70!black},
	bodyfont=\normalfont,
	mdframed={
		linewidth=2pt,
		rightline=false, topline=false, bottomline=false,
		linecolor=WildStrawberry, backgroundcolor=WildStrawberry!5,
		nobreak=false
	}
]{thmpinkbox}

% Purple box (Conjectures)
\declaretheoremstyle[
	headfont=\bfseries\sffamily\color{Violet!70!black},
	bodyfont=\normalfont,
	mdframed={
		linewidth=2pt,
		rightline=false, topline=false, bottomline=false,
		linecolor=Violet, backgroundcolor=Violet!1,
		nobreak=false
	}
]{conjpurplebox}

% Gray box (Notes, Notation)
\declaretheoremstyle[
	headfont=\bfseries\sffamily\color{Gray!70!black},
	bodyfont=\normalfont,
	mdframed={
		linewidth=2pt,
		rightline=false, topline=false, bottomline=false,
		linecolor=Gray, backgroundcolor=Gray!5,
		nobreak=false
	}
]{notgraybox}

\declaretheoremstyle[
	headfont=\bfseries\sffamily\color{Gray!70!black},
	bodyfont=\normalfont,
	mdframed={
		linewidth=2pt,
		rightline=false, topline=false, bottomline=false,
		linecolor=Gray,
		nobreak=false
	}
]{notgrayline}

% Explanation box
\declaretheoremstyle[
	headfont=\bfseries\sffamily\color{NavyBlue!70!black},
	bodyfont=\normalfont,
	numbered=no,
	mdframed={
		linewidth=2pt,
		rightline=false, topline=false, bottomline=false,
		linecolor=NavyBlue, backgroundcolor=NavyBlue!1,
		nobreak=false
	}
]{thmexplanationbox}

% Answer box
\declaretheoremstyle[
	headfont=\bfseries\sffamily\color{WildStrawberry!70!black},
	bodyfont=\normalfont,
	numbered=no,
	mdframed={
		linewidth=2pt,
		rightline=false, topline=false, bottomline=false,
		linecolor=WildStrawberry, backgroundcolor=WildStrawberry!1,
		nobreak=false
	}
]{thmanswerbox}

% ==================== THEOREM DECLARATIONS ====================
\declaretheorem[style=thmgreenbox, name=Definition, numberwithin=section]{definition}
\declaretheorem[style=thmgreen2box, name=Definition, numbered=no]{definition*}
\declaretheorem[style=thmredbox, name=Theorem, numberwithin=section]{theorem}
\declaretheorem[style=thmred2box, name=Theorem, numbered=no]{theorem*}
\declaretheorem[style=thmredbox, name=Lemma, numberwithin=section]{lemma}
\declaretheorem[style=thmredbox, name=Proposition, numberwithin=section]{proposition}
\declaretheorem[style=thmredbox, name=Corollary, numberwithin=section]{corollary}
\declaretheorem[style=thmpinkbox, name=Problem, numberwithin=section]{problem}
\declaretheorem[style=thmpinkbox, name=Problem, numbered=no]{problem*}
\declaretheorem[style=thmblue2box, name=Claim, numbered=no]{claim}
\declaretheorem[style=conjpurplebox, name=Conjecture, numberwithin=section]{conjecture}
\declaretheorem[style=thmbluebox, numbered=no, name=Example]{eg}
\declaretheorem[style=thmblueline, numbered=no, name=Remark]{remark}
\declaretheorem[style=thmblueline, numbered=no, name=Note]{note}
\declaretheorem[style=thmpinkbox, numbered=no, name=Exercise]{exercise}
\declaretheorem[style=notgrayline, numbered=no, name=As previously seen]{prev}
\declaretheorem[style=thmgreen3box, numbered=no, name=Intuition]{intuition}
\declaretheorem[style=notgraybox, numbered=no, name=Notation]{notation}

% ==================== HYPERREF FIXES ====================
\renewcommand\theHdefinition{\thesection.\arabic{definition}}
\renewcommand\theHtheorem{\thesection.\arabic{theorem}}
\renewcommand\theHlemma{\thesection.\arabic{lemma}}
\renewcommand\theHproposition{\thesection.\arabic{proposition}}
\renewcommand\theHcorollary{\thesection.\arabic{corollary}}
\renewcommand\theHproblem{\thesection.\arabic{problem}}
\renewcommand\theHconjecture{\thesection.\arabic{conjecture}}

% ==================== CUSTOM PROOF ENVIRONMENT ====================
\makeatletter
\renewenvironment{proof}[1][\proofname]{\par
	\pushQED{\qed}%
	\normalfont \topsep-2\p@\@plus6\p@\relax
	\trivlist
	\item[\hskip\labelsep
		\color{RawSienna!70!black}\sffamily\bfseries
		#1\@addpunct{.}]\ignorespaces
	\begin{mdframed}[linewidth=2pt,rightline=false, topline=false, bottomline=false,linecolor=RawSienna, backgroundcolor=RawSienna!1]
}{%
	\popQED\endtrivlist\@endpefalse
	\end{mdframed}
}
\makeatother

\renewcommand{\qed}{\null\hfill\(\blacksquare\)}

% ==================== EXPLANATION ENVIRONMENT ====================
\declaretheorem[style=thmexplanationbox, numbered=no, name=Proof]{tmpexplanation}
\newenvironment{explanation}[1][]{\vspace{-10pt}\pushQED{\(\circledast\)}\begin{tmpexplanation}}{\null\hfill\popQED\end{tmpexplanation}}

% ==================== ANSWER ENVIRONMENT ====================
\declaretheorem[style=thmanswerbox, numbered=no, name=Answer]{tmpanswer}
\newenvironment{answer}[1][]{\vspace{-10pt}\pushQED{\(\circledast\)}\begin{tmpanswer}}{\null\hfill\popQED\end{tmpanswer}}

% ==================== UTILITY COMMANDS ====================
% Loop through and input unit files
\newcommand{\units}[2]{%
	\foreach \c in {#1,...,#2}{%
		\IfFileExists{Units/unit_\c.tex}{%
			\input{Units/unit_\c.tex}%
			\clearpage
		}{}
	}
}
